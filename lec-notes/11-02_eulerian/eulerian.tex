\documentclass{article}

\usepackage[margin=1in]{geometry}

\usepackage{algorithm,algorithmic}
\renewcommand{\algorithmicrequire}{\textbf{Input:}}
\renewcommand{\algorithmicensure}{\textbf{Output:}}
\renewcommand{\algorithmiccomment}[2]{\hspace{#1}$\triangleright$ {#2} \hfill }

\usepackage{amsfonts}
\def\R{{\mathbb R}}
\def\N{{\mathbb N}}

\title{Topological Sort: Decrementing Functions and Loop Invariants}
\author{CSCI 432}

\begin{document}
\maketitle

\noindent
Name:\\
Who did you work with today?

\section*{Eulerian Tour: The Algorithm}

The following algorithm takes as input a graph $G=(V,E)$ with exactly two
vertices with odd degree, and returns a sequence
of vertices that define the Eulerian tour.

\begin{algorithm}[h]
    \caption{FindEulerianTour}
    %
    \begin{algorithmic}[1]
        \REQUIRE a connected graph $G=(V,E)$ such that exactly two vertices have odd
        degree

        \ENSURE $tour$, a list of $|E|+1$ vertices in G in a total order compatible with
        the partial order of the DAG.

        \STATE $s,t \gets$ the two vertices of odd degree

        \FOR {$e \in E$}
            \STATE Add an attribute `$e$.visited' and initialize it to \textsc{False}
            \STATE Add an attribute `$e$.next' and initialize it to \textsc{Null}
            \STATE Add an attribute `$e$.prev' and initialize it to \textsc{Null}
        \ENDFOR

        \STATE $w \gets$ any neighbor of $s$

        \STATE Use DFS to find a path from $s$ to $t$, starting with $(s,w)$,
        using the `next' and `prev' to store the next edge in the path,  and marking all
        edges $e$ in the path as $e$.visited=\textsc{True}.

        \WHILE {$\exists e \in E$ such that $e.visited=$\textsc{False}}
            \STATE $v\gets$ a vertex that already has been seen along the
                current path from $s$ to $t$ and has an outgoing edge $(v,w) \in
                E$ such that $(v,w)$.visited is \textsc{False}
            \STATE $e_1=(v,x) \gets $ a visited edge leaving $v$
            \STATE $e_2 \gets e_1.$prev
            \STATE $e_2$.next $\gets (v,w)$
            \STATE $(v,w).$prev $\gets e_2$
            \STATE $(v,w)$.visited $\gets$ \textsc{True}
            \STATE Follow DFS from $w$, updated `next', `prev', and `visited' as
                appropriate until reaching $v$ again, and using $e_1$ as the
                `next' edge of the last edge in that new path
        \ENDWHILE

        \STATE $L \gets$ a list of vertices, initialized to $L=[s]$
        \STATE $(x,y) \gets (s,w)$
        \WHILE {$y \neq t$}
            \STATE Append $y$ to $L$
            \STATE $(x,y) \gets (x,y)$.next
        \ENDWHILE
        \STATE Append $t$ to $L$

        \RETURN $L$
    \end{algorithmic}
\end{algorithm}

\pagebreak

Walk through a small example.

\textbf{Answer}

\pagebreak
\section*{Eulerian Tour: The Decrementing Function}

What is the decrementing function for the first while loop?

\textbf{Answer}
\vspace{1in}


Can you prove that this function is well-defined and decrementing?

\textbf{Answer}
\vspace{1in}


\pagebreak
\section*{Eulerian Tour: The Loop Invariant}

Here, we investigate partial correctness of the first while loop.  For this loop, what
are the following statements (start with your best guess, then come back and
revise if needed):

The loop guard $G$:
\vspace{0.5in}

The post-condition $Q$:
\vspace{0.5in}

The pre-condition $P$:
\vspace{0.5in}

The loop invariant $L=L_i$:
\vspace{0.5in}


Can you use this loop invariant
to prove partial correctness (Remember, there are three parts to partial
correctness: Initialization, Maintenance, and End).

\textbf{Answer}
\vspace{1in}

\end{document}
