\documentclass{article}
\usepackage{../fasy-hw}

%% UPDATE these variables:
\renewcommand{\hwnum}{3}
\title{Advanced Algorithms, Homework \hwnum}
\author{\todo{Your Name Here}}
\collab{n/a}
\date{due: 7 October 2022}

\begin{document}

\maketitle

This homework assignment should be
submitted as a single PDF file to D2L.

General homework expectations:
\begin{itemize}
    \item Homework should be typeset using LaTex.
    \item Answers should be in complete sentences and proofread.
    \item You will not plagiarize, nor will you share your written solutions
        with classmates.
    \item List collaborators at the start of each question using the
        \texttt{collab} command.
    \item Put your answers where the \texttt{todo} command currently is (and
        remove the \texttt{todo}, but not the word \texttt{Answer}).
    \item If you are asked to come up with an algorithm, you are
        expected to give an algorithm that beats the brute force (and, if possible, of
        optimal time complexity). With your algorithm, please provide the following:
        \begin{itemize}
            \item \emph{What}: A prose explanation of the problem and the algorithm,
                including a description of the input/output.
            \item \emph{How}: Describe how the algorithm works, including giving
                psuedocode for it.  Be sure to reference the pseudocode
                from within the prose explanation.
            \item \emph{How Fast}: Runtime, along with justification.  (Or, at
                the very least, a proof of termination using a decrementing function).
           % \item \emph{Why}: Briefly explain why the algorithm works.  Be sure
           %     to include a statement of the loop invariant for each loop, or
           %     recursion invariant for each recursive function.
        \end{itemize}

\end{itemize}


%%%%%%%%%%%%%%%%%%%%%%%%%%%%%%%%%%%%%%%%%%%%%%%%%%%%%%%%%%%%%%%%%%%%%%%%%%%%%%
\nextprob{}
Consider the solution to the Towers of Hanoi Problem (See Figure~$1.4$ in the
textbook or EPI 15.1).  Prove that the recursion terminates by providing the
decrementing function for the recursion.

\paragraph{Answer}

\todo{replace this TODO with your answer}
%%%%%%%%%%%%%%%%%%%%%%%%%%%%%%%%%%%%%%%%%%%%%%%%%%%%%%%%%%%%%%%%%%%%%%%%%%%%%%

%%%%%%%%%%%%%%%%%%%%%%%%%%%%%%%%%%%%%%%%%%%%%%%%%%%%%%%%%%%%%%%%%%%%%%%%%%%%%%
\collab{\todo{}}
\nextprob{Merge Sorted Arrays}

Give a linear-time algorithm that takes two sorted arrays of real numbers as
input, and returns a merged list of sorted numbers.  You should give your answer
in pseudocode.  For \emph{How Fast}, please provide both the decrementing
function of any loop or recursion, as well as a proof of the linear runtime.

\paragraph{Answer}

\todo{replace this TODO with your answer}
%%%%%%%%%%%%%%%%%%%%%%%%%%%%%%%%%%%%%%%%%%%%%%%%%%%%%%%%%%%%%%%%%%%%%%%%%%%%%%

%%%%%%%%%%%%%%%%%%%%%%%%%%%%%%%%%%%%%%%%%%%%%%%%%%%%%%%%%%%%%%%%%%%%%%%%%%%%%%
\collab{\todo{}}
\nextprob{TODO}

Chapter 2, Problem 1b (Generalized \textsc{SubsetSum}).

\paragraph{Answer}

\todo{replace this TODO with your answer}
%%%%%%%%%%%%%%%%%%%%%%%%%%%%%%%%%%%%%%%%%%%%%%%%%%%%%%%%%%%%%%%%%%%%%%%%%%%%%%

\end{document}
