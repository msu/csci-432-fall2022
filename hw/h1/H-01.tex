\documentclass{article}
\usepackage{../fasy-hw}

%% UPDATE these variables:
\renewcommand{\hwnum}{1}
\title{Advanced Algorithms, Homework \hwnum}
\author{\todo{Your Name Here}}
\collab{\todo{list your collaborators here}}
\date{due: Monday, 29 August 2021}

\begin{document}

\maketitle

This homework assignment should be
submitted as a single PDF file both to D2L and to Gradescope.

General homework expectations:
\begin{itemize}
    \item Homework should be typeset using LaTex.
    \item Answers should be in complete sentences and proofread.
    \item You will not plagiarize, nor will you share your written solutions
        with classmates.
    \item List collaborators at the start of each question using the
        \texttt{collab} command.
    \item Put your answers where the \texttt{todo} command currently is (and
        remove the \texttt{todo}, but not the word \texttt{Answer}).
\end{itemize}

%%%%%%%%%%%%%%%%%%%%%%%%%%%%%%%%%%%%%%%%%%%%%%%%%%%%%%%%%%%%%%%%%%%%%%%%%%%%%%
\collab{\todo{}}
\nextprob{Sorting}
Consider your favorite sorting algorithm.
\begin{enumerate}
    \item \emph{What} is the problem that this algorithm solves?

        \paragraph{Answer}
        \todo{}

    \item \emph{How} does it work? (You are required to describe at a high level
        how it works.  For a bonus, also give the pseudocode!)

        \paragraph{Answer}
        \todo{}

    \item \emph{How fast} does it work?  Give the asymptotic running time.
        Note: typically, you will give this as the worst-case running time.
        However, if you chose quicksort or another randomized algorithm, please
        give both the worst-case running time and the expected running time.  No
        justification of the running time is needed.

        \paragraph{Answer}
        \todo{}

    \item \emph{How much space} does it require?  Give the asymptotic storage
        complexity.  Briefly justify.

        \paragraph{Answer}
        \todo{}

    \item \emph{Why} does this work? Typically, this will be given as a loop
        invariant proof.  For this HW, explain why it works informally, in your
        own words.

        \paragraph{Answer}
        \todo{}

\end{enumerate}

Note: in all future HWs, if you are asked to come up with an algorithm, you are
expected to give an algorithm that beats the brute force (and, if possible, of
optimal time complexity). With your algorithm, please provide the following:
\begin{itemize}
    \item \emph{What}: A prose explanation of the problem and the algorithm,
        including a description of the input/output.
    \item \emph{How}: Psuedocode, referenced from within the prose explanation.
    \item \emph{How Fast}: Runtime, along with justification.  (Or, in the
        extreme, a proof of termination).
    \item \emph{Why}: Statement of the loop invariant for each loop.
\end{itemize}
Providing the storage complexity is always welcome, but only required when
explicitly asked for.

\end{document}
