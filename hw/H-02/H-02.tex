\documentclass{article}
\usepackage{../fasy-hw}

%% UPDATE these variables:
\renewcommand{\hwnum}{2}
\title{Advanced Algorithms, Homework \hwnum}
\author{\todo{Your Name Here}}
\collab{n/a}
\date{due: 28 September 2022}

\begin{document}

\maketitle

This homework assignment should be
submitted as a single PDF file to D2L.

General homework expectations:
\begin{itemize}
    \item Homework should be typeset using LaTex.
    \item Answers should be in complete sentences and proofread.
    \item You will not plagiarize, nor will you share your written solutions
        with classmates.
    \item List collaborators at the start of each question using the
        \texttt{collab} command.
    \item Put your answers where the \texttt{todo} command currently is (and
        remove the \texttt{todo}, but not the word \texttt{Answer}).
\end{itemize}

%%%%%%%%%%%%%%%%%%%%%%%%%%%%%%%%%%%%%%%%%%%%%%%%%%%%%%%%%%%%%%%%%%%%%%%%%%%%%%
\collab{\todo{}}
\nextprob{Redo!}
You've seen this problem before, and have feedback from the prior submission.
Even if you earned a high pass (HP) on this problem, be sure to carefully read
through and improve upon your previous submission.  Expectations are higher on a
redo than on the first submission.

Prove using induction that the closed form of:
$$T(n) = \begin{cases}
            1        & n=1\\
            T(n-1)+n & n>1
         \end{cases}
$$
is $O(n^2)$.

\paragraph{Answer}

\todo{replace this TODO with your answer}
%%%%%%%%%%%%%%%%%%%%%%%%%%%%%%%%%%%%%%%%%%%%%%%%%%%%%%%%%%%%%%%%%%%%%%%%%%%%%%

%%%%%%%%%%%%%%%%%%%%%%%%%%%%%%%%%%%%%%%%%%%%%%%%%%%%%%%%%%%%%%%%%%%%%%%%%%%%%%
\collab{\todo{}}
\nextprob{Closed Form of Recurrence Relations}

What is the closed form of the following recurrence relation:
$$ T(n) = \begin{cases}
            1        & n=1\\
            3T(n-1)+2 & n>1
         \end{cases}
$$
Explain how you found the closed form, and prove that it is correct.

\paragraph{Answer}

\todo{replace this TODO with your answer}

%%%%%%%%%%%%%%%%%%%%%%%%%%%%%%%%%%%%%%%%%%%%%%%%%%%%%%%%%%%%%%%%%%%%%%%%%%%%%%

%%%%%%%%%%%%%%%%%%%%%%%%%%%%%%%%%%%%%%%%%%%%%%%%%%%%%%%%%%%%%%%%%%%%%%%%%%%%%%
\collab{\todo{}}
\nextprob{Solving Recurrence Relations in Asymptotic Form}

What is the asymptotic form of the following recurrence relations?  Use Master's
theorem to justify your answers:
\begin{enumerate}
    \item $T(n) = 16 T(n/4) + \Theta(n)$
    \item $T(n) = 2 T(n/2) + n \log{n}$
    \item $T(n) = 6 T(n/3) + n^2 \log{n}$
    \item $T(n) = 4 T(n/2) + n^2$
    \item $T(n) = 9 T(n/3) + n$
\end{enumerate}
Note: we assume that $T(1)=\Theta(1)$ whenever it is not explicitly given.

\paragraph{Answer}

\todo{replace this TODO with your answer}
%%%%%%%%%%%%%%%%%%%%%%%%%%%%%%%%%%%%%%%%%%%%%%%%%%%%%%%%%%%%%%%%%%%%%%%%%%%%%%

\end{document}
