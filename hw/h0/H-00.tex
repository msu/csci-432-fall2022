\documentclass{article}
\usepackage{../fasy-hw}

%% UPDATE these variables:
\renewcommand{\hwnum}{0}
\title{Advanced Algorithms, Homework \hwnum}
\author{\todo{Your Name Here}}
\collab{\todo{list your collaborators here}}
\date{due: Monday, 28 August 2021}

\begin{document}

\maketitle

This homework assignment should be
submitted as a single PDF file both to D2L and to Gradescope.

General homework expectations:
\begin{itemize}
    \item Homework should be typeset using LaTex.
    \item Answers should be in complete sentences and proofread.
    \item You will not plagiarize, nor will you share your written solutions
        with classmates.
    \item List collaborators at the start of each question using the
        \texttt{collab} command.
    \item Put your answers where the \texttt{todo} command currently is (and
        remove the \texttt{todo}, but not the word \texttt{Answer}).
\end{itemize}

%%%%%%%%%%%%%%%%%%%%%%%%%%%%%%%%%%%%%%%%%%%%%%%%%%%%%%%%%%%%%%%%%%%%%%%%%%%%%%
\collab{N/A}
\nextprob{Getting to Know You}

Answer the following questions:
\begin{enumerate}
    \item What is your elevator pitch?  Describe yourself in 1-2
                sentences.

        \paragraph{Answer} \todo{}

     \item What was your favorite CS class so far, and why?

         \paragraph{Answer} \todo{}

     \item What was your least favorite CS class so far, and why?

         \paragraph{Answer} \todo{}
     \item Why are you interested in taking this course?

         \paragraph{Answer} \todo{}

     \item What is your biggest academic or research goal for this semester (can
         be related to this course or not)?

         \paragraph{Answer} \todo{}

     \item What do you want to do after you graduate?

         \paragraph{Answer} \todo{}

     \item What was the most challenging aspect of your coursework last semester?

         \paragraph{Answer} \todo{}

    \item Is your photo in D2L a recognizable photo of yourself?  (Note: the
        answer should be ``Yes'').

         \paragraph{Answer} \todo{}

\end{enumerate}

%%%%%%%%%%%%%%%%%%%%%%%%%%%%%%%%%%%%%%%%%%%%%%%%%%%%%%%%%%%%%%%%%%%%%%%%%%%%%%
\collab{\todo{}}
\nextprob{Plagiarism}

\begin{enumerate}

    \item In this class, please properly cite all resources that you use. Please
        write your own definition of plagiarism here.  Remember to cite all
        sources!

        \paragraph{Answer}
        \todo{}

    \item If you have observed plagiarism or cheating in a classroom (either as
        an instructor or as a student), explain the situation and how it made
        you feel.  If you have not experienced plagiarism or cheating or if you
        would prefer not to reflect on a personal experience, find a news
        article about plagiarism or cheating and explain how you would feel if
        you were one of the people involved.

        \paragraph{Answer}
        \todo{}
\end{enumerate}

%%%%%%%%%%%%%%%%%%%%%%%%%%%%%%%%%%%%%%%%%%%%%%%%%%%%%%%%%%%%%%%%%%%%%%%%%%%%%%
\nextprob{}
Prove using induction that the closed form of:
$$T(n) = \begin{cases}
            1        & n=1\\
            T(n-1)+n & n>1
         \end{cases}
$$
is $O(n^2)$.


%%%%%%%%%%%%%%%%%%%%%%%%%%%%%%%%%%%%%%%%%%%%%%%%%%%%%%%%%%%%%%%%%%%%%%%%%%%%%%
\collab{\todo{}}
\nextprob{Big-O}
Prove that $f(x)=n^2 + 3n -2$ is
$\Theta(n^2)$ (be sure to use the definition!).

\paragraph{Answer}

\todo{}

%%%%%%%%%%%%%%%%%%%%%%%%%%%%%%%%%%%%%%%%%%%%%%%%%%%%%%%%%%%%%%%%%%%%%%%%%%%%%%
\collab{\todo{}}
\nextprob{If/Then Statements}
Consider the following statement: If $a$ and $b$ are both even numbers, then $ab$ is
an even number.  Let's call this statement $P$.
\begin{enumerate}
    \item What is the definition of an odd number?

        \paragraph{Answer}
        \todo{}

    \item What is the definition of an even number?

        \paragraph{Answer}
        \todo{}

    \item What is the contrapositive of $P$?

        \paragraph{Answer}
        \todo{}

    \item What is the converse of $P$?

        \paragraph{Answer}
        \todo{}

    \item Prove $P$.

        \paragraph{Answer}
        \todo{}

\end{enumerate}



%%%%%%%%%%%%%%%%%%%%%%%%%%%%%%%%%%%%%%%%%%%%%%%%%%%%%%%%%%%%%%%%%%%%%%%%%%%%%%
\collab{\todo{}}
\nextprob{Sorting}
Consider your favorite sorting algorithm.
\begin{enumerate}
    \item \emph{What} is the problem that this algorithm solves?

        \paragraph{Answer}
        \todo{}

    \item \emph{How} does it work? (You are required to describe at a high level
        how it works.  For a bonus, also give the pseudocode!)

        \paragraph{Answer}
        \todo{}

    \item \emph{How fast} does it work?  Give the asymptotic running time.
        Note: typically, you will give this as the worst-case running time.
        However, if you chose quicksort or another randomized algorithm, please
        give both the worst-case running time and the expected running time.  No
        justification of the running time is needed.

        \paragraph{Answer}
        \todo{}

    \item \emph{Why} does this work? Typically, this will be given as a loop
        invariant proof.  For this HW, explain why it works informally, in your
        own words.

        \paragraph{Answer}
        \todo{}

\end{enumerate}

Note: in all future HWs, if you are asked to come up with an algorithm, you are
expected to give an algorithm that beats the brute force (and, if possible, of
optimal time complexity). With your algorithm, please provide the following:
\begin{itemize}
    \item \emph{What}: A prose explanation of the problem and the algorithm,
        including a description of the input/output.
    \item \emph{How}: Psuedocode, referenced from within the prose explanation.
    \item \emph{How Fast}: Runtime, along with justification.  (Or, in the
        extreme, a proof of termination).
    \item \emph{Why}: Statement of the loop invariant for each loop.
\end{itemize}


\end{document}
